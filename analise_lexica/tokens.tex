\section{Especificação de tokens}

\begin{frame}[fragile]{Alfabetos}

    \begin{block}{Definição de alfabeto}
        Um alfabeto, ou classe de caracteres, é um conjunto finito de símbolos.
    \end{block}
    \pause

    \vspace{0.2in}

    Exemplos de alfabetos: ASCII, EBCDIC, a alfabeto binário $\{\ 0, 1\ \}$, os dígitos decimais, etc.

\end{frame}

\begin{frame}[fragile]{Cadeias}

    \begin{block}{Definição de cadeia}
        Uma cadeia sobre um alfabeto $\mathcal{A}$ é uma sequência finita de elementos de $\mathcal{A}$. Os termos sentença, palavra e string são geralmente
        usados como sinônimos de cadeia.
    \end{block}

\end{frame}

\begin{frame}[fragile]{Conceitos associados à cadeias}

    \begin{itemize}
        \item O comprimento (número de caracteres) de uma cadeia $s$ é denotado por $|s|$
        \pause

        \item A cadeia vazia \mintinline{apl}{∊} tem comprimento igual a zero
        \pause

        \item Um prefixo de $s$ é uma cadeia obtida pela remoção de zero ou mais caracteres do fim de $s$
        \pause

        \item Um sufixo de $s$ é uma cadeia obtida pela remoção de zero ou mais caracteres do início de $s$
        \pause

        \item Uma subcadeia de $s$ é uma cadeia obtida pela remoção de um prefixo e de um sufixo de $s$
        \pause

        \item Um prefixo, sufixo ou subcadeia de $s$ são ditos próprios se diferem de \mintinline{apl}{∊} e de $s$
        \pause

        \item Um subsequência de $s$ é uma cadeia obtida pela remoção de zero ou mais símbolos de $s$, não necessariamente contíguos
    \end{itemize}

\end{frame}

\begin{frame}[fragile]{Linguagens}

    \begin{block}{Definição de linguagem}
        Uma linguagem é um conjunto de cadeias sobre algum alfabeto $\mathcal{A}$ fixo.
    \end{block}
    \pause

    \vspace{0.2in}
    Esta definição contempla também linguagens abstratas como $\emptyset$ (o conjunto vazio), ou $\{$ \code{apl}{∊} $\}$, o conjunto contendo apenas a cadeia vazia.
\end{frame}

\begin{frame}[fragile]{Operações em cadeias}

    \begin{itemize}
        \item Se $x$ e $y$ são duas cadeias, então a concatenação de $x$ e $y$, denotada $xy$, é a cadeia formada pelo acréscimo, ao final de $x$, de todos 
            os caracteres de $y$, na mesma ordem
        \pause

        \item Por exemplo, se $x = \code{cpp}{"rodo"}$ e $y = \code{cpp}{"via"}$, então $xy = \code{cpp}{"rodovia"}$
        \pause

        \item A cadeia vazia \code{apl}{∊} é o elemento neutro da concatenação
        \pause

        \item Se a concatenação for visualizada como um produto, é possível definir uma ``exponenciação'' de cadeias
        \pause

        \item Seja $s$ uma cadeia e $n$ um natural. Então
        \pause
        \begin{enumerate}
            \item $s^0 = \code{apl}{∊}$
            \pause

            \item $s^n = ss^{n - 1}$
        \end{enumerate}
    
    \end{itemize}

\end{frame}

\begin{frame}[fragile]{Operações em linguagens}

    Sejam $L$ e $M$ duas linguagens. São definidas as seguintes operações sobre linguagens:
    \pause
    \vspace{0.2in}

    \begin{tabularx}{0.95\textwidth}{llX}
        \toprule
        \textbf{Operação} & \textbf{Notação} & \textbf{Definição} \\
        \midrule
        união & $L\cup M$ & $L\cup M = \{\ s\ |\ s\in L\ \vee\ s\in M\ \}$ \\
        \rowcolor[gray]{0.9}
        concatenação & $LM$ & $LM = \{\ st\ |\ s\in L\ \land\ t\in M\ \}$ \\
        fechamento de Kleene & $\displaystyle L^*$ & $\displaystyle L^* = \bigcup_{i = 0}^\infty L^i$ \\
        \rowcolor[gray]{0.9}
        fechamento positivo & $L^+$ & $\displaystyle L^+ = \bigcup_{i = 1}^\infty L^i$ \\
        \bottomrule
    \end{tabularx}

\end{frame}

\begin{frame}[fragile]{Exemplos de operações em linguagens}

    Seja $L = \{$ \texttt{A, B, C,} $\ldots$ \texttt{Z, a, b, c,} $\ldots$ \texttt{z} $\}$ e $M = \{$ \texttt{0, 1, 2,} $\ldots$ \texttt{9} $\}$. Então:
    \vspace{0.2in}
    \pause

    \begin{enumerate}
        \item $L\cup M$ é o conjunto de letras e dígitos
        \pause

        \item $LM$ é o conjunto de cadeias formadas por uma letra, seguida de um dígito
        \pause

        \item $L^4$ é o conjunto de todas as cadeias formadas por exatamente quatro letras
        \pause

        \item $L^*$ é o conjunto de todas as cadeias formadas por letras, incluíndo a cadeia \code{apl}{∊}
        \pause

        \item $L(L\cup D)^*$ é o conjunto de cadeias de letras e dígitos, que iniciam com uma letra
        \pause

        \item $D^+$ é o conjunto de cadeias formadas por um ou mais dígitos
    \end{enumerate}

\end{frame}

\begin{frame}[fragile]{Expressões regulares}

    \begin{block}{Definição de expressão regular}
        Sejam $\Sigma$ um alfabeto. As expressões regulares sobre $\Sigma$ são definidas pelas seguintes regras, onde cada expressão regular define uma linguagem:
        \begin{enumerate}
            \item \code{apl}{∊} é uma expressão regular que denota a linguagem $\{$ \code{apl}{∊} $\}$
            \item Se $a\in\Sigma$, então $a$ é uma expressão regular que denota a linguagem $\{\ a\ \}$
            \item Se $r$ e $s$ são duas expressões regulares que denotam as linguagens $L(r)$ e  $L(s)$, então
            \begin{enumerate}[(a)]
                \item $(r)$ é uma expressão regular que denota $L(r)$
                \item $(r)|(s)$ é uma expressão regular que denota $L(r)\cup L(s)$
                \item $(r)(s)$ é uma expressão regular que denota $L(r)L(s)$
                \item $(r)^*$ é uma expressão regular que denota $(L(r))^*$
            \end{enumerate}
        \end{enumerate}
    \end{block}

\end{frame}

\begin{frame}[fragile]{Expressões regulares e parêntesis}

    O uso de parêntesis em expressões regulares pode ser reduzido se forem adotadas as seguintes convenções:
    \pause

    \vspace{0.2in}

    \begin{enumerate}
        \item o operador unário $*$ possui a maior precedência e é associativo à esquerda
        \pause

        \item a concatenação tem a segunda maior precedência e é associativa à esquerda
        \pause

        \item o operador $|$ tem a menor precedência e é associativo à esquerda
        \pause
    \end{enumerate}

    \vspace{0.2in}

    Neste cenário, a expressão regular $(a)\ |\ ((b)^*\ (c))$ equivale a $a\ |\ b^*\ c$.

\end{frame}

\begin{frame}[fragile]{Exemplos de expressões regulares}

    Seja $\Sigma = \{\ a, b\ \}$. Então
    \pause

    \vspace{0.2in}
    \begin{itemize}
        \item $a\ |\ b$ denota a linguagem $\{\ a, b\ \}$
        \pause

        \item $(a\ |\ b)(a\ |\ b)$ denota $\{\ aa, ab, ba, bb\ \}$
        \pause

        \item $a^*$ denota $\{\ \code{apl}{∊}, a, aa, aaa, \ldots\ \}$
        \pause

        \item $(a\ |\ b)^*$ denota todas as cadeias formas por zero ou mais instâncias de $a$ ou de $b$
        \pause

        \item $a\ |\ a^*\ b$ denota a cadeia $a$ e todas as cadeias iniciadas por zero ou mais $a$'s, seguidos de um $b$
    \end{itemize}

\end{frame}

\begin{frame}[fragile]{Propriedades das expressões regulares}

    Sejam $r, s, t$ expressões regulares. Valem as seguintes propriedades:
    \vspace{0.2in}

    \begin{tabularx}{0.95\textwidth}{cX}
        \toprule
        \textbf{Axioma} & \textbf{Descrição}\\
        \midrule
        $r|s = s|r$ & $|$ é comutativo \\
        \rowcolor[gray]{0.9}
        $r|(s|t) = (r|s)|t$ & $|$ é associativo \\
        $r(st) = (rs)t$ & a concatenação é associativa \\
        \rowcolor[gray]{0.9}
        $\begin{array}{l}r(s|t) = rs|rt\\ (r|s)t = rt|st\end{array}$ & {a concatenação é distributiva em relação a $|$} \\
        $\begin{array}{l}\code{apl}{∊}r = r\\ r\code{apl}{∊} = r\end{array}$ & \code{apl}{∊} é o elemento neutro da concatenação \\
        \rowcolor[gray]{0.9}
        $r^* = (r|\code{apl}{∊})^*$ & relação entre \code{apl}{∊} e $*$ \\
        $r^{**} = r^*$ & $*$ é idempotente \\
        \bottomrule
    \end{tabularx}

\end{frame}

\begin{frame}[fragile]{Definições regulares}

    \begin{block}{Definição}
        Seja $\Sigma$ um alfabeto. Uma definição regular sobre $\Sigma$ é uma sequência de definições da forma
        \[
            \begin{array}{l}
                d_1 \to r_1 \\
                d_2 \to r_2 \\
                \ldots \\
                d_n \to r_n
            \end{array}
        \]
    onde cada $d_i$ é um nome distinto e $r_i$ uma expressão regular sobre o alfabeto
    $\Sigma \cup \{\ d_1, d_2, \ldots, d_{i - 1}\ \}$.
    \end{block}

\end{frame}

\begin{frame}[fragile]{Exemplo de definição regular}

    Os identificadores de Pascal, e em muitas outras linguagens, são formados por cadeias de caracteres e dígitos, começando com uma letra.
    \pause

    \vspace{0.2in}
    Abaixo segue a definição regular para o conjunto de todos os identificadores válidos em Pascal:
    \pause

    \vspace{0.2in}
    \[
        \begin{array}{rl}
            \mathbf{letra}\to & \texttt{A | B | ... | Z | a | b | ... | z} \\
            \mathbf{digito}\to & \texttt{0 | 1 | 2 | ... | 9} \\
            \mathbf{id}\to & \mathbf{letra}\ |\ (\mathbf{letra}\ |\ \mathbf{digito})^* 
        \end{array}
    \]

\end{frame}

\begin{frame}[fragile]{Simplificações notacionais}

    As seguintes notações podem simplificar as expressões regulares:
    \pause

    \vspace{0.2in}

    \begin{enumerate}
        \item \textit{Uma ou mais ocorrências}. Se $r$ é uma expressão regular, então $(r)^+$ denota $(L(r))^+$. O operador $+$ tem a mesma associatividade e
            precedência do operator $*$. Vale que $r^* = r^+|\code{apl}{∊}$ e que $r^+ = rr^*$.
        \pause

        \item \textit{Zero ou mais ocorrências}. Se $r$ é uma expressão regular, então $r?$ denota $L(r)\cup \code{apl}{∊}$. O operador $?$ é posfixo e unário, 
        e $r? = r|\code{apl}{∊}$.
        \pause

        \item \textit{Classes de caracteres}. A notação \texttt{[}$abc$\texttt{]}, onde $a, b, c$ são símbolos do alfabeto, denota a expressão regular 
            $a\ |\ b\ |\ c$. A notação \texttt{[}$a$\texttt{-}$z$\texttt{]} abrevia a expressão regular $a\ |\ b\ |\ \ldots\ |\ z$.
    \end{enumerate}

\end{frame}

\begin{frame}[fragile]{Limitações das expressões regulares}

    \begin{itemize}
        \item Existem linguagens que não podem ser descritas por meio de expressões regulares
        \pause

        \item Por exemplo, não é possível descrever o conjunto $\mathcal{P}$ de todas as cadeias de parêntesis balanceados por meio de expressões regulares
        \pause

        \item Contudo, o conjunto $\mathcal{P}$ pode ser descrito por meio de uma gramática livre de contexto
        \pause

        \item Existem linguagens que não podem ser descritas nem mesmo por meio de uma gramática livre de contexto
        \pause

        \item Por exemplo, o conjunto 
        \[
            \mathcal{C} = \{ wcw\ |\ w\ \mbox{é uma cadeia de $a$'s e $b$'s} \}
        \]
        não pode ser descrito nem por expressões regulares e nem por meio de uma gramática livre de contexto
    \end{itemize}

\end{frame}
