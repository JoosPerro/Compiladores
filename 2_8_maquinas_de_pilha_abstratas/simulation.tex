\begin{frame}[fragile]{Exemplo de avaliação da expressão \code{cpp}{12+3*} por máquina abstrata de pilha}

    \begin{figure}
        \centering 
        \begin{tikzpicture}
            \node[draw,opacity=0] at (0, 0) { };
            \node[draw,opacity=0] at (12, 6) { };

            \node (A) at (1, 5) { \textbf{Instrução} };
            %\node (A1) at (1.5, 4) { \code{cpp}{1} };
            %\node (A2) at (1.8, 4) { \code{cpp}{2} };
            %\node (A3) at (2.1, 4) { \code{cpp}{+} };
            %\node (A4) at (2.4, 4) { \code{cpp}{3} };
            %\node (A5) at (2.7, 4) { \code{cpp}{*} };

            %\draw[thick,-latex] (1.5, 3.2) to (A1);


            \node[anchor=west] at (7, 5) { \textbf{Ações} };
            %\node[anchor=west] at (8, 4) { \footnotesize \texttt{Empilhar o valor \code{cpp}{1}} };

            \node[anchor=west] at (3, 1) { \textbf{Pilha} };
            \draw[very thick] (5, 0) to (7, 0);

            %\draw[thick] (5.5, 0) rectangle (6.5, 1);
            %\node at (6, 0.5) { \code{cpp}{1} };

        \end{tikzpicture}
    \end{figure}

\end{frame}

\begin{frame}[fragile]{Exemplo de avaliação da expressão \code{cpp}{12+3*} por máquina abstrata de pilha}

    \begin{figure}
        \centering 
        \begin{tikzpicture}
            \node[draw,opacity=0] at (0, 0) { };
            \node[draw,opacity=0] at (12, 6) { };

            \node (A) at (1, 5) { \textbf{Instrução} };
            \node (A1) at (1.5, 4) { \code{cpp}{1} };
            \node (A2) at (1.8, 4) { \code{cpp}{2} };
            \node (A3) at (2.1, 4) { \code{cpp}{+} };
            \node (A4) at (2.4, 4) { \code{cpp}{3} };
            \node (A5) at (2.7, 4) { \code{cpp}{*} };

            \draw[thick,-latex] (1.5, 3.2) to (A1);


            \node[anchor=west] at (7, 5) { \textbf{Ações} };
            %\node[anchor=west] at (8, 4) { \footnotesize \texttt{Empilhar o valor \code{cpp}{1}} };

            \node[anchor=west] at (3, 1) { \textbf{Pilha} };
            \draw[very thick] (5, 0) to (7, 0);

            %\draw[thick] (5.5, 0) rectangle (6.5, 1);
            %\node at (6, 0.5) { \code{cpp}{1} };

        \end{tikzpicture}
    \end{figure}

\end{frame}

\begin{frame}[fragile]{Exemplo de avaliação da expressão \code{cpp}{12+3*} por máquina abstrata de pilha}

    \begin{figure}
        \centering 
        \begin{tikzpicture}
            \node[draw,opacity=0] at (0, 0) { };
            \node[draw,opacity=0] at (12, 6) { };

            \node (A) at (1, 5) { \textbf{Instrução} };
            \node (A1) at (1.5, 4) { \code{cpp}{1} };
            \node (A2) at (1.8, 4) { \code{cpp}{2} };
            \node (A3) at (2.1, 4) { \code{cpp}{+} };
            \node (A4) at (2.4, 4) { \code{cpp}{3} };
            \node (A5) at (2.7, 4) { \code{cpp}{*} };

            \draw[thick,-latex] (1.5, 3.2) to (A1);


            \node[anchor=west] at (7, 5) { \textbf{Ações} };
            \node[anchor=west] at (8, 4) { \footnotesize \texttt{Empilhar o valor \code{cpp}{1}} };

            \node[anchor=west] at (3, 1) { \textbf{Pilha} };
            \draw[very thick] (5, 0) to (7, 0);

            %\draw[thick] (5.5, 0) rectangle (6.5, 1);
            %\node at (6, 0.5) { \code{cpp}{1} };

        \end{tikzpicture}
    \end{figure}

\end{frame}

\begin{frame}[fragile]{Exemplo de avaliação da expressão \code{cpp}{12+3*} por máquina abstrata de pilha}

    \begin{figure}
        \centering 
        \begin{tikzpicture}
            \node[draw,opacity=0] at (0, 0) { };
            \node[draw,opacity=0] at (12, 6) { };

            \node (A) at (1, 5) { \textbf{Instrução} };
            \node (A1) at (1.5, 4) { \code{cpp}{1} };
            \node (A2) at (1.8, 4) { \code{cpp}{2} };
            \node (A3) at (2.1, 4) { \code{cpp}{+} };
            \node (A4) at (2.4, 4) { \code{cpp}{3} };
            \node (A5) at (2.7, 4) { \code{cpp}{*} };

            \draw[thick,-latex] (1.5, 3.2) to (A1);


            \node[anchor=west] at (7, 5) { \textbf{Ações} };
            \node[anchor=west] at (8, 4) { \footnotesize \texttt{Empilhar o valor \code{cpp}{1}} };

            \node[anchor=west] at (3, 1) { \textbf{Pilha} };
            \draw[very thick] (5, 0) to (7, 0);

            \draw[thick] (5.5, 0) rectangle (6.5, 1);
            \node at (6, 0.5) { \code{cpp}{1} };

        \end{tikzpicture}
    \end{figure}

\end{frame}

\begin{frame}[fragile]{Exemplo de avaliação da expressão \code{cpp}{12+3*} por máquina abstrata de pilha}

    \begin{figure}
        \centering 
        \begin{tikzpicture}
            \node[draw,opacity=0] at (0, 0) { };
            \node[draw,opacity=0] at (12, 6) { };

            \node (A) at (1, 5) { \textbf{Instrução} };
            \node (A1) at (1.5, 4) { \code{cpp}{1} };
            \node (A2) at (1.8, 4) { \code{cpp}{2} };
            \node (A3) at (2.1, 4) { \code{cpp}{+} };
            \node (A4) at (2.4, 4) { \code{cpp}{3} };
            \node (A5) at (2.7, 4) { \code{cpp}{*} };

            \draw[thick,-latex] (1.8, 3.2) to (A2);


            \node[anchor=west] at (7, 5) { \textbf{Ações} };
%            \node[anchor=west] at (8, 4) { \footnotesize \texttt{Empilhar o valor \code{cpp}{1}} };

            \node[anchor=west] at (3, 1) { \textbf{Pilha} };
            \draw[very thick] (5, 0) to (7, 0);

            \draw[thick] (5.5, 0) rectangle (6.5, 1);
            \node at (6, 0.5) { \code{cpp}{1} };

        \end{tikzpicture}
    \end{figure}

\end{frame}

\begin{frame}[fragile]{Exemplo de avaliação da expressão \code{cpp}{12+3*} por máquina abstrata de pilha}

    \begin{figure}
        \centering 
        \begin{tikzpicture}
            \node[draw,opacity=0] at (0, 0) { };
            \node[draw,opacity=0] at (12, 6) { };

            \node (A) at (1, 5) { \textbf{Instrução} };
            \node (A1) at (1.5, 4) { \code{cpp}{1} };
            \node (A2) at (1.8, 4) { \code{cpp}{2} };
            \node (A3) at (2.1, 4) { \code{cpp}{+} };
            \node (A4) at (2.4, 4) { \code{cpp}{3} };
            \node (A5) at (2.7, 4) { \code{cpp}{*} };

            \draw[thick,-latex] (1.8, 3.2) to (A2);


            \node[anchor=west] at (7, 5) { \textbf{Ações} };
            \node[anchor=west] at (8, 4) { \footnotesize \texttt{Empilhar o valor \code{cpp}{2}} };

            \node[anchor=west] at (3, 1) { \textbf{Pilha} };
            \draw[very thick] (5, 0) to (7, 0);

            \draw[thick] (5.5, 0) rectangle (6.5, 1);
            \node at (6, 0.5) { \code{cpp}{1} };

        \end{tikzpicture}
    \end{figure}

\end{frame}

\begin{frame}[fragile]{Exemplo de avaliação da expressão \code{cpp}{12+3*} por máquina abstrata de pilha}

    \begin{figure}
        \centering 
        \begin{tikzpicture}
            \node[draw,opacity=0] at (0, 0) { };
            \node[draw,opacity=0] at (12, 6) { };

            \node (A) at (1, 5) { \textbf{Instrução} };
            \node (A1) at (1.5, 4) { \code{cpp}{1} };
            \node (A2) at (1.8, 4) { \code{cpp}{2} };
            \node (A3) at (2.1, 4) { \code{cpp}{+} };
            \node (A4) at (2.4, 4) { \code{cpp}{3} };
            \node (A5) at (2.7, 4) { \code{cpp}{*} };

            \draw[thick,-latex] (1.8, 3.2) to (A2);


            \node[anchor=west] at (7, 5) { \textbf{Ações} };
            \node[anchor=west] at (8, 4) { \footnotesize \texttt{Empilhar o valor \code{cpp}{2}} };

            \node[anchor=west] at (3, 1) { \textbf{Pilha} };
            \draw[very thick] (5, 0) to (7, 0);

            \draw[thick] (5.5, 0) rectangle (6.5, 1);
            \node at (6, 0.5) { \code{cpp}{1} };

            \draw[thick] (5.5, 1) rectangle (6.5, 2);
            \node at (6, 1.5) { \code{cpp}{2} };

        \end{tikzpicture}
    \end{figure}

\end{frame}

\begin{frame}[fragile]{Exemplo de avaliação da expressão \code{cpp}{12+3*} por máquina abstrata de pilha}

    \begin{figure}
        \centering 
        \begin{tikzpicture}
            \node[draw,opacity=0] at (0, 0) { };
            \node[draw,opacity=0] at (12, 6) { };

            \node (A) at (1, 5) { \textbf{Instrução} };
            \node (A1) at (1.5, 4) { \code{cpp}{1} };
            \node (A2) at (1.8, 4) { \code{cpp}{2} };
            \node (A3) at (2.1, 4) { \code{cpp}{+} };
            \node (A4) at (2.4, 4) { \code{cpp}{3} };
            \node (A5) at (2.7, 4) { \code{cpp}{*} };

            \draw[thick,-latex] (2.1, 3.2) to (A3);


            \node[anchor=west] at (7, 5) { \textbf{Ações} };
            %\node[anchor=west] at (8, 4) { \footnotesize \texttt{Empilhar o valor \code{cpp}{2}} };

            \node[anchor=west] at (3, 1) { \textbf{Pilha} };
            \draw[very thick] (5, 0) to (7, 0);

            \draw[thick] (5.5, 0) rectangle (6.5, 1);
            \node at (6, 0.5) { \code{cpp}{1} };

            \draw[thick] (5.5, 1) rectangle (6.5, 2);
            \node at (6, 1.5) { \code{cpp}{2} };

        \end{tikzpicture}
    \end{figure}

\end{frame}

\begin{frame}[fragile]{Exemplo de avaliação da expressão \code{cpp}{12+3*} por máquina abstrata de pilha}

    \begin{figure}
        \centering 
        \begin{tikzpicture}
            \node[draw,opacity=0] at (0, 0) { };
            \node[draw,opacity=0] at (12, 6) { };

            \node (A) at (1, 5) { \textbf{Instrução} };
            \node (A1) at (1.5, 4) { \code{cpp}{1} };
            \node (A2) at (1.8, 4) { \code{cpp}{2} };
            \node (A3) at (2.1, 4) { \code{cpp}{+} };
            \node (A4) at (2.4, 4) { \code{cpp}{3} };
            \node (A5) at (2.7, 4) { \code{cpp}{*} };

            \draw[thick,-latex] (2.1, 3.2) to (A3);


            \node[anchor=west] at (7, 5) { \textbf{Ações} };
            \node[anchor=west] at (8, 4) { \footnotesize \texttt{Adicionar \code{cpp}{1+2} } };
%            \node[anchor=west] at (8, 3.7) { \footnotesize \texttt{Inserir a soma no } };
%            \node[anchor=west] at (8, 3.4) { \footnotesize \texttt{topo da pilha } };

            \node[anchor=west] at (3, 1) { \textbf{Pilha} };
            \draw[very thick] (5, 0) to (7, 0);

            \draw[thick] (5.5, 0) rectangle (6.5, 1);
            \node at (6, 0.5) { \code{cpp}{1} };

            \draw[thick] (5.5, 1) rectangle (6.5, 2);
            \node at (6, 1.5) { \code{cpp}{2} };

        \end{tikzpicture}
    \end{figure}

\end{frame}

\begin{frame}[fragile]{Exemplo de avaliação da expressão \code{cpp}{12+3*} por máquina abstrata de pilha}

    \begin{figure}
        \centering 
        \begin{tikzpicture}
            \node[draw,opacity=0] at (0, 0) { };
            \node[draw,opacity=0] at (12, 6) { };

            \node (A) at (1, 5) { \textbf{Instrução} };
            \node (A1) at (1.5, 4) { \code{cpp}{1} };
            \node (A2) at (1.8, 4) { \code{cpp}{2} };
            \node (A3) at (2.1, 4) { \code{cpp}{+} };
            \node (A4) at (2.4, 4) { \code{cpp}{3} };
            \node (A5) at (2.7, 4) { \code{cpp}{*} };

            \draw[thick,-latex] (2.1, 3.2) to (A3);


            \node[anchor=west] at (7, 5) { \textbf{Ações} };
            \node[anchor=west] at (8, 4) { \footnotesize \texttt{Adicionar \code{cpp}{1+2} } };
            \node[anchor=west] at (8, 3.5) { \footnotesize \texttt{Empilhar a soma} };

            \node[anchor=west] at (3, 1) { \textbf{Pilha} };
            \draw[very thick] (5, 0) to (7, 0);

            \draw[thick] (5.5, 0) rectangle (6.5, 1);
            \node at (6, 0.5) { \code{cpp}{1} };

            \draw[thick] (5.5, 1) rectangle (6.5, 2);
            \node at (6, 1.5) { \code{cpp}{2} };

        \end{tikzpicture}
    \end{figure}

\end{frame}

\begin{frame}[fragile]{Exemplo de avaliação da expressão \code{cpp}{12+3*} por máquina abstrata de pilha}

    \begin{figure}
        \centering 
        \begin{tikzpicture}
            \node[draw,opacity=0] at (0, 0) { };
            \node[draw,opacity=0] at (12, 6) { };

            \node (A) at (1, 5) { \textbf{Instrução} };
            \node (A1) at (1.5, 4) { \code{cpp}{1} };
            \node (A2) at (1.8, 4) { \code{cpp}{2} };
            \node (A3) at (2.1, 4) { \code{cpp}{+} };
            \node (A4) at (2.4, 4) { \code{cpp}{3} };
            \node (A5) at (2.7, 4) { \code{cpp}{*} };

            \draw[thick,-latex] (2.1, 3.2) to (A3);


            \node[anchor=west] at (7, 5) { \textbf{Ações} };
            \node[anchor=west] at (8, 4) { \footnotesize \texttt{Adicionar \code{cpp}{1+2} } };
            \node[anchor=west] at (8, 3.5) { \footnotesize \texttt{Empilhar a soma} };

            \node[anchor=west] at (3, 1) { \textbf{Pilha} };
            \draw[very thick] (5, 0) to (7, 0);

            \draw[thick] (5.5, 0) rectangle (6.5, 1);
            \node at (6, 0.5) { \code{cpp}{3} };

%            \draw[thick] (5.5, 1) rectangle (6.5, 2);
%            \node at (6, 1.5) { \code{cpp}{2} };

        \end{tikzpicture}
    \end{figure}

\end{frame}

\begin{frame}[fragile]{Exemplo de avaliação da expressão \code{cpp}{12+3*} por máquina abstrata de pilha}

    \begin{figure}
        \centering 
        \begin{tikzpicture}
            \node[draw,opacity=0] at (0, 0) { };
            \node[draw,opacity=0] at (12, 6) { };

            \node (A) at (1, 5) { \textbf{Instrução} };
            \node (A1) at (1.5, 4) { \code{cpp}{1} };
            \node (A2) at (1.8, 4) { \code{cpp}{2} };
            \node (A3) at (2.1, 4) { \code{cpp}{+} };
            \node (A4) at (2.4, 4) { \code{cpp}{3} };
            \node (A5) at (2.7, 4) { \code{cpp}{*} };

            \draw[thick,-latex] (2.4, 3.2) to (A4);


            \node[anchor=west] at (7, 5) { \textbf{Ações} };
            \node[anchor=west] at (8, 4) { \footnotesize \texttt{Empilhar o valor 3} };
%            \node[anchor=west] at (8, 3.7) { \footnotesize \texttt{Inserir a soma no } };
%            \node[anchor=west] at (8, 3.4) { \footnotesize \texttt{topo da pilha } };

            \node[anchor=west] at (3, 1) { \textbf{Pilha} };
            \draw[very thick] (5, 0) to (7, 0);

            \draw[thick] (5.5, 0) rectangle (6.5, 1);
            \node at (6, 0.5) { \code{cpp}{3} };

%            \draw[thick] (5.5, 1) rectangle (6.5, 2);
%            \node at (6, 1.5) { \code{cpp}{2} };

        \end{tikzpicture}
    \end{figure}

\end{frame}

\begin{frame}[fragile]{Exemplo de avaliação da expressão \code{cpp}{12+3*} por máquina abstrata de pilha}

    \begin{figure}
        \centering 
        \begin{tikzpicture}
            \node[draw,opacity=0] at (0, 0) { };
            \node[draw,opacity=0] at (12, 6) { };

            \node (A) at (1, 5) { \textbf{Instrução} };
            \node (A1) at (1.5, 4) { \code{cpp}{1} };
            \node (A2) at (1.8, 4) { \code{cpp}{2} };
            \node (A3) at (2.1, 4) { \code{cpp}{+} };
            \node (A4) at (2.4, 4) { \code{cpp}{3} };
            \node (A5) at (2.7, 4) { \code{cpp}{*} };

            \draw[thick,-latex] (2.4, 3.2) to (A4);


            \node[anchor=west] at (7, 5) { \textbf{Ações} };
            \node[anchor=west] at (8, 4) { \footnotesize \texttt{Empilhar o valor 3} };
%            \node[anchor=west] at (8, 3.7) { \footnotesize \texttt{Inserir a soma no } };
%            \node[anchor=west] at (8, 3.4) { \footnotesize \texttt{topo da pilha } };

            \node[anchor=west] at (3, 1) { \textbf{Pilha} };
            \draw[very thick] (5, 0) to (7, 0);

            \draw[thick] (5.5, 0) rectangle (6.5, 1);
            \node at (6, 0.5) { \code{cpp}{3} };

            \draw[thick] (5.5, 1) rectangle (6.5, 2);
            \node at (6, 1.5) { \code{cpp}{3} };

        \end{tikzpicture}
    \end{figure}

\end{frame}

\begin{frame}[fragile]{Exemplo de avaliação da expressão \code{cpp}{12+3*} por máquina abstrata de pilha}

    \begin{figure}
        \centering 
        \begin{tikzpicture}
            \node[draw,opacity=0] at (0, 0) { };
            \node[draw,opacity=0] at (12, 6) { };

            \node (A) at (1, 5) { \textbf{Instrução} };
            \node (A1) at (1.5, 4) { \code{cpp}{1} };
            \node (A2) at (1.8, 4) { \code{cpp}{2} };
            \node (A3) at (2.1, 4) { \code{cpp}{+} };
            \node (A4) at (2.4, 4) { \code{cpp}{3} };
            \node (A5) at (2.7, 4) { \code{cpp}{*} };

            \draw[thick,-latex] (2.7, 3.2) to (A5);


            \node[anchor=west] at (7, 5) { \textbf{Ações} };
%            \node[anchor=west] at (8, 4) { \footnotesize \texttt{Empilhar o valor 3} };
%            \node[anchor=west] at (8, 3.7) { \footnotesize \texttt{Inserir a soma no } };
%            \node[anchor=west] at (8, 3.4) { \footnotesize \texttt{topo da pilha } };

            \node[anchor=west] at (3, 1) { \textbf{Pilha} };
            \draw[very thick] (5, 0) to (7, 0);

            \draw[thick] (5.5, 0) rectangle (6.5, 1);
            \node at (6, 0.5) { \code{cpp}{3} };

            \draw[thick] (5.5, 1) rectangle (6.5, 2);
            \node at (6, 1.5) { \code{cpp}{3} };

        \end{tikzpicture}
    \end{figure}

\end{frame}

\begin{frame}[fragile]{Exemplo de avaliação da expressão \code{cpp}{12+3*} por máquina abstrata de pilha}

    \begin{figure}
        \centering 
        \begin{tikzpicture}
            \node[draw,opacity=0] at (0, 0) { };
            \node[draw,opacity=0] at (12, 6) { };

            \node (A) at (1, 5) { \textbf{Instrução} };
            \node (A1) at (1.5, 4) { \code{cpp}{1} };
            \node (A2) at (1.8, 4) { \code{cpp}{2} };
            \node (A3) at (2.1, 4) { \code{cpp}{+} };
            \node (A4) at (2.4, 4) { \code{cpp}{3} };
            \node (A5) at (2.7, 4) { \code{cpp}{*} };

            \draw[thick,-latex] (2.7, 3.2) to (A5);


            \node[anchor=west] at (7, 5) { \textbf{Ações} };
            \node[anchor=west] at (8, 4) { \footnotesize \texttt{Multiplicar \code{cpp}{3*3} } };
%            \node[anchor=west] at (8, 4) { \footnotesize \texttt{Empilhar o valor 3} };

            \node[anchor=west] at (3, 1) { \textbf{Pilha} };
            \draw[very thick] (5, 0) to (7, 0);

            \draw[thick] (5.5, 0) rectangle (6.5, 1);
            \node at (6, 0.5) { \code{cpp}{3} };

            \draw[thick] (5.5, 1) rectangle (6.5, 2);
            \node at (6, 1.5) { \code{cpp}{3} };

        \end{tikzpicture}
    \end{figure}

\end{frame}

\begin{frame}[fragile]{Exemplo de avaliação da expressão \code{cpp}{12+3*} por máquina abstrata de pilha}

    \begin{figure}
        \centering 
        \begin{tikzpicture}
            \node[draw,opacity=0] at (0, 0) { };
            \node[draw,opacity=0] at (12, 6) { };

            \node (A) at (1, 5) { \textbf{Instrução} };
            \node (A1) at (1.5, 4) { \code{cpp}{1} };
            \node (A2) at (1.8, 4) { \code{cpp}{2} };
            \node (A3) at (2.1, 4) { \code{cpp}{+} };
            \node (A4) at (2.4, 4) { \code{cpp}{3} };
            \node (A5) at (2.7, 4) { \code{cpp}{*} };

            \draw[thick,-latex] (2.7, 3.2) to (A5);


            \node[anchor=west] at (7, 5) { \textbf{Ações} };
            \node[anchor=west] at (8, 4) { \footnotesize \texttt{Multiplicar \code{cpp}{3*3} } };
            \node[anchor=west] at (8, 3.5) { \footnotesize \texttt{Empilhar o produto} };

            \node[anchor=west] at (3, 1) { \textbf{Pilha} };
            \draw[very thick] (5, 0) to (7, 0);

            \draw[thick] (5.5, 0) rectangle (6.5, 1);
            \node at (6, 0.5) { \code{cpp}{3} };

            \draw[thick] (5.5, 1) rectangle (6.5, 2);
            \node at (6, 1.5) { \code{cpp}{3} };

        \end{tikzpicture}
    \end{figure}

\end{frame}

\begin{frame}[fragile]{Exemplo de avaliação da expressão \code{cpp}{12+3*} por máquina abstrata de pilha}

    \begin{figure}
        \centering 
        \begin{tikzpicture}
            \node[draw,opacity=0] at (0, 0) { };
            \node[draw,opacity=0] at (12, 6) { };

            \node (A) at (1, 5) { \textbf{Instrução} };
            \node (A1) at (1.5, 4) { \code{cpp}{1} };
            \node (A2) at (1.8, 4) { \code{cpp}{2} };
            \node (A3) at (2.1, 4) { \code{cpp}{+} };
            \node (A4) at (2.4, 4) { \code{cpp}{3} };
            \node (A5) at (2.7, 4) { \code{cpp}{*} };

            \draw[thick,-latex] (2.7, 3.2) to (A5);


            \node[anchor=west] at (7, 5) { \textbf{Ações} };
            \node[anchor=west] at (8, 4) { \footnotesize \texttt{Multiplicar \code{cpp}{3*3} } };
            \node[anchor=west] at (8, 3.5) { \footnotesize \texttt{Empilhar o produto} };

            \node[anchor=west] at (3, 1) { \textbf{Pilha} };
            \draw[very thick] (5, 0) to (7, 0);

            \draw[thick] (5.5, 0) rectangle (6.5, 1);
            \node at (6, 0.5) { \code{cpp}{9} };

%            \draw[thick] (5.5, 1) rectangle (6.5, 2);
%            \node at (6, 1.5) { \code{cpp}{3} };

        \end{tikzpicture}
    \end{figure}

\end{frame}
