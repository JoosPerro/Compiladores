\section{Máquinas de Pilha Abstratas}

\begin{frame}[fragile]{Máquinas de Pilha Abstratas}

    \begin{itemize}
        \item A interface de vanguarda do compilador produz uma representação intermediária do programa fonte, que será usada pela interface de retaguarda para
            produzir o programa alvo
        \pause

        \item Uma possível forma para a representação intermediária é a máquina de pilha abstrata
        \pause

        \item Uma máquina de pilha abstrata possui memórias separadas para dados e instruções, e todas as operações aritméticas são realizadas sobre os valores
            em uma pilha
        \pause

        \item As instruções são divididas em três classes: aritmética inteira, manipulação de pilha e fluxo de controle
        \pause

        \item O ponteiro $pc$ indica qual é a próxima instrução a ser executada
    \end{itemize}

\end{frame}

\begin{frame}[fragile]{Instruções aritméticas}

    \begin{itemize}
        \item  A máquina de pilha abstrata precisa implementar cada operador da linguagem intermediária
        \pause

        \item Operações elementares, como adição e subtração, são suportadas diretamente
        \pause

        \item Operações mais sofisticadas devem ser implementadas como uma sequência de instruções da máquina
        \pause

        \item A título de simplificação, assuma que existe uma instrução para cada operação aritmética
        \pause

        \item O código de uma máquina de pilha abstrata para uma expressão simula a avaliação de um representação posfixa, usando uma pilha
        \pause

        \item A avaliação segue da esquerda para a direita, empilhando os operandos
        \pause

        \item  Quando um operador é encontrado, seus operandos são extraídos da pilha
        (do último para o primeiro), a operação é realizada e o resultado é inserido no topo da pilha
    \end{itemize}

\end{frame}

\begin{frame}[fragile]{Comportamento do analisador preditivo não-recursivo para a cadeia ``$\textbf{id}\code{apl}{+}\textbf{id}\code{apl}{×}\textbf{id}$''}

    \begin{tikzpicture}
        \node[opacity=0] at (12, 0) { . };
        \node[opacity=0] at (12, 7) { . };

        \node at (2.5, 0.5) { \texttt{Pilha} };

        \draw[very thick] (1.5, 1) to (3.5, 1);

        \draw[thick] (2, 1) rectangle (3, 2);
        \node at (2.5, 1.5) { \texttt{\$} };

        \draw[thick] (2, 2) rectangle (3, 3);
        \node at (2.5, 2.5) { $E$ };

        \node[anchor=west] at (5, 5) { \texttt{Entrada} };

        \node[anchor=west] at (6, 4) { \textbf{id} \code{apl}{+} \textbf{id} \code{apl}{×} \textbf{id} \texttt{\$} };

        \node[anchor=west] at (10, 5) { \texttt{Saída} };

%        \node[anchor=west] at (11, 4) { $E'\to \code{apl}{+}TE'$ };
    \end{tikzpicture}

\end{frame}

\begin{frame}[fragile]{Comportamento do analisador preditivo não-recursivo para a cadeia ``$\textbf{id}\code{apl}{+}\textbf{id}\code{apl}{×}\textbf{id}$''}

    \begin{tikzpicture}
        \node[opacity=0] at (12, 0) { . };
        \node[opacity=0] at (12, 7) { . };

        \node at (2.5, 0.5) { \texttt{Pilha} };

        \draw[very thick] (1.5, 1) to (3.5, 1);

        \draw[thick] (2, 1) rectangle (3, 2);
        \node at (2.5, 1.5) { \texttt{\$} };

        \draw[thick] (2, 2) rectangle (3, 3);
        \node at (2.5, 2.5) { $E$ };

        \node[anchor=west] at (5, 5) { \texttt{Entrada} };

        \node[anchor=west] at (6, 4) { \textbf{id} \code{apl}{+} \textbf{id} \code{apl}{×} \textbf{id} \texttt{\$} };
        \draw[-latex,thick] (6.3, 3.2) to (6.3, 3.8);

        \node[anchor=west] at (10, 5) { \texttt{Saída} };

%        \node[anchor=west] at (11, 4) { $E'\to \code{apl}{+}TE'$ };
    \end{tikzpicture}

\end{frame}

\begin{frame}[fragile]{Comportamento do analisador preditivo não-recursivo para a cadeia ``$\textbf{id}\code{apl}{+}\textbf{id}\code{apl}{×}\textbf{id}$''}

    \begin{tikzpicture}
        \node[opacity=0] at (12, 0) { . };
        \node[opacity=0] at (12, 7) { . };

        \node at (2.5, 0.5) { \texttt{Pilha} };

        \draw[very thick] (1.5, 1) to (3.5, 1);

        \draw[thick] (2, 1) rectangle (3, 2);
        \node at (2.5, 1.5) { \texttt{\$} };

        \draw[thick] (2, 2) rectangle (3, 3);
        \node at (2.5, 2.5) { $E'$ };

        \draw[thick] (2, 3) rectangle (3, 4);
        \node at (2.5, 3.5) { $T$ };

        \node[anchor=west] at (5, 5) { \texttt{Entrada} };

        \node[anchor=west] at (6, 4) { \textbf{id} \code{apl}{+} \textbf{id} \code{apl}{×} \textbf{id} \texttt{\$} };
        \draw[-latex,thick] (6.3, 3.2) to (6.3, 3.8);

        \node[anchor=west] at (10, 5) { \texttt{Saída} };

        \node[anchor=west] at (11, 4) { $E\to TE'$ };
%        \node[anchor=west] at (11, 4) { $E'\to \code{apl}{+}TE'$ };
    \end{tikzpicture}

\end{frame}

\begin{frame}[fragile]{Comportamento do analisador preditivo não-recursivo para a cadeia ``$\textbf{id}\code{apl}{+}\textbf{id}\code{apl}{×}\textbf{id}$''}

    \begin{tikzpicture}
        \node[opacity=0] at (12, 0) { . };
        \node[opacity=0] at (12, 7) { . };

        \node at (2.5, 0.5) { \texttt{Pilha} };

        \draw[very thick] (1.5, 1) to (3.5, 1);

        \draw[thick] (2, 1) rectangle (3, 2);
        \node at (2.5, 1.5) { \texttt{\$} };

        \draw[thick] (2, 2) rectangle (3, 3);
        \node at (2.5, 2.5) { $E'$ };

        \draw[thick] (2, 3) rectangle (3, 4);
        \node at (2.5, 3.5) { $T'$ };

        \draw[thick] (2, 4) rectangle (3, 5);
        \node at (2.5, 4.5) { $F$ };

        \node[anchor=west] at (5, 5) { \texttt{Entrada} };

        \node[anchor=west] at (6, 4) { \textbf{id} \code{apl}{+} \textbf{id} \code{apl}{×} \textbf{id} \texttt{\$} };
        \draw[-latex,thick] (6.3, 3.2) to (6.3, 3.8);

        \node[anchor=west] at (10, 5) { \texttt{Saída} };

        \node[anchor=west] at (11, 4) { $T\to FT'$ };
%        \node[anchor=west] at (11, 4) { $E'\to \code{apl}{+}TE'$ };
    \end{tikzpicture}

\end{frame}

\begin{frame}[fragile]{Comportamento do analisador preditivo não-recursivo para a cadeia ``$\textbf{id}\code{apl}{+}\textbf{id}\code{apl}{×}\textbf{id}$''}

    \begin{tikzpicture}
        \node[opacity=0] at (12, 0) { . };
        \node[opacity=0] at (12, 7) { . };

        \node at (2.5, 0.5) { \texttt{Pilha} };

        \draw[very thick] (1.5, 1) to (3.5, 1);

        \draw[thick] (2, 1) rectangle (3, 2);
        \node at (2.5, 1.5) { \texttt{\$} };

        \draw[thick] (2, 2) rectangle (3, 3);
        \node at (2.5, 2.5) { $E'$ };

        \draw[thick] (2, 3) rectangle (3, 4);
        \node at (2.5, 3.5) { $T'$ };

        \draw[thick] (2, 4) rectangle (3, 5);
        \node at (2.5, 4.5) { \textbf{id} };

        \node[anchor=west] at (5, 5) { \texttt{Entrada} };

        \node[anchor=west] at (6, 4) { \textbf{id} \code{apl}{+} \textbf{id} \code{apl}{×} \textbf{id} \texttt{\$} };
        \draw[-latex,thick] (6.3, 3.2) to (6.3, 3.8);

        \node[anchor=west] at (10, 5) { \texttt{Saída} };

        \node[anchor=west] at (11, 4) { $F\to \textbf{id}$ };
%        \node[anchor=west] at (11, 4) { $E'\to \code{apl}{+}TE'$ };
    \end{tikzpicture}

\end{frame}

\begin{frame}[fragile]{Comportamento do analisador preditivo não-recursivo para a cadeia ``$\textbf{id}\code{apl}{+}\textbf{id}\code{apl}{×}\textbf{id}$''}

    \begin{tikzpicture}
        \node[opacity=0] at (12, 0) { . };
        \node[opacity=0] at (12, 7) { . };

        \node at (2.5, 0.5) { \texttt{Pilha} };

        \draw[very thick] (1.5, 1) to (3.5, 1);

        \draw[thick] (2, 1) rectangle (3, 2);
        \node at (2.5, 1.5) { \texttt{\$} };

        \draw[thick] (2, 2) rectangle (3, 3);
        \node at (2.5, 2.5) { $E'$ };

        \draw[thick] (2, 3) rectangle (3, 4);
        \node at (2.5, 3.5) { $T'$ };

%        \draw[thick] (2, 4) rectangle (3, 5);
%        \node at (2.5, 4.5) { \textbf{id} };

        \node[anchor=west] at (5, 5) { \texttt{Entrada} };

        \node[anchor=west] at (6, 4) { \textbf{id} \code{apl}{+} \textbf{id} \code{apl}{×} \textbf{id} \texttt{\$} };
        \draw[-latex,thick] (6.675, 3.2) to (6.675, 3.8);

        \node[anchor=west] at (10, 5) { \texttt{Saída} };

%        \node[anchor=west] at (11, 4) { $F\to \textbf{id}$ };
%        \node[anchor=west] at (11, 4) { $E'\to \code{apl}{+}TE'$ };
    \end{tikzpicture}

\end{frame}

\begin{frame}[fragile]{Comportamento do analisador preditivo não-recursivo para a cadeia ``$\textbf{id}\code{apl}{+}\textbf{id}\code{apl}{×}\textbf{id}$''}

    \begin{tikzpicture}
        \node[opacity=0] at (12, 0) { . };
        \node[opacity=0] at (12, 7) { . };

        \node at (2.5, 0.5) { \texttt{Pilha} };

        \draw[very thick] (1.5, 1) to (3.5, 1);

        \draw[thick] (2, 1) rectangle (3, 2);
        \node at (2.5, 1.5) { \texttt{\$} };

        \draw[thick] (2, 2) rectangle (3, 3);
        \node at (2.5, 2.5) { $E'$ };

%        \draw[thick] (2, 3) rectangle (3, 4);
%        \node at (2.5, 3.5) { $T'$ };

%        \draw[thick] (2, 4) rectangle (3, 5);
%        \node at (2.5, 4.5) { \textbf{id} };

        \node[anchor=west] at (5, 5) { \texttt{Entrada} };

        \node[anchor=west] at (6, 4) { \textbf{id} \code{apl}{+} \textbf{id} \code{apl}{×} \textbf{id} \texttt{\$} };
        \draw[-latex,thick] (6.675, 3.2) to (6.675, 3.8);

        \node[anchor=west] at (10, 5) { \texttt{Saída} };

        \node[anchor=west] at (11, 4) { $T'\to \code{apl}{∊}$ };
%        \node[anchor=west] at (11, 4) { $E'\to \code{apl}{+}TE'$ };
    \end{tikzpicture}

\end{frame}

\begin{frame}[fragile]{Comportamento do analisador preditivo não-recursivo para a cadeia ``$\textbf{id}\code{apl}{+}\textbf{id}\code{apl}{×}\textbf{id}$''}

    \begin{tikzpicture}
        \node[opacity=0] at (12, 0) { . };
        \node[opacity=0] at (12, 7) { . };

        \node at (2.5, 0.5) { \texttt{Pilha} };

        \draw[very thick] (1.5, 1) to (3.5, 1);

        \draw[thick] (2, 1) rectangle (3, 2);
        \node at (2.5, 1.5) { \texttt{\$} };

        \draw[thick] (2, 2) rectangle (3, 3);
        \node at (2.5, 2.5) { $E'$ };

        \draw[thick] (2, 3) rectangle (3, 4);
        \node at (2.5, 3.5) { $T$ };

        \draw[thick] (2, 4) rectangle (3, 5);
        \node at (2.5, 4.5) { \code{apl}{+} };

        \node[anchor=west] at (5, 5) { \texttt{Entrada} };

        \node[anchor=west] at (6, 4) { \textbf{id} \code{apl}{+} \textbf{id} \code{apl}{×} \textbf{id} \texttt{\$} };
        \draw[-latex,thick] (6.675, 3.2) to (6.675, 3.8);

        \node[anchor=west] at (10, 5) { \texttt{Saída} };

        \node[anchor=west] at (11, 4) { $E'\to \code{apl}{+}TE'$ };
    \end{tikzpicture}

\end{frame}

\begin{frame}[fragile]{Comportamento do analisador preditivo não-recursivo para a cadeia ``$\textbf{id}\code{apl}{+}\textbf{id}\code{apl}{×}\textbf{id}$''}

    \begin{tikzpicture}
        \node[opacity=0] at (12, 0) { . };
        \node[opacity=0] at (12, 7) { . };

        \node at (2.5, 0.5) { \texttt{Pilha} };

        \draw[very thick] (1.5, 1) to (3.5, 1);

        \draw[thick] (2, 1) rectangle (3, 2);
        \node at (2.5, 1.5) { \texttt{\$} };

        \draw[thick] (2, 2) rectangle (3, 3);
        \node at (2.5, 2.5) { $E'$ };

        \draw[thick] (2, 3) rectangle (3, 4);
        \node at (2.5, 3.5) { $T$ };

%        \draw[thick] (2, 4) rectangle (3, 5);
%        \node at (2.5, 4.5) { \code{apl}{+} };

        \node[anchor=west] at (5, 5) { \texttt{Entrada} };

        \node[anchor=west] at (6, 4) { \textbf{id} \code{apl}{+} \textbf{id} \code{apl}{×} \textbf{id} \texttt{\$} };
        \draw[-latex,thick] (7.05, 3.2) to (7.05, 3.8);

        \node[anchor=west] at (10, 5) { \texttt{Saída} };

%        \node[anchor=west] at (11, 4) { $E'\to \code{apl}{+}TE'$ };
    \end{tikzpicture}

\end{frame}

\begin{frame}[fragile]{Comportamento do analisador preditivo não-recursivo para a cadeia ``$\textbf{id}\code{apl}{+}\textbf{id}\code{apl}{×}\textbf{id}$''}

    \begin{tikzpicture}
        \node[opacity=0] at (12, 0) { . };
        \node[opacity=0] at (12, 7) { . };

        \node at (2.5, 0.5) { \texttt{Pilha} };

        \draw[very thick] (1.5, 1) to (3.5, 1);

        \draw[thick] (2, 1) rectangle (3, 2);
        \node at (2.5, 1.5) { \texttt{\$} };

        \draw[thick] (2, 2) rectangle (3, 3);
        \node at (2.5, 2.5) { $E'$ };

        \draw[thick] (2, 3) rectangle (3, 4);
        \node at (2.5, 3.5) { $T'$ };

        \draw[thick] (2, 4) rectangle (3, 5);
        \node at (2.5, 4.5) { $F$ };

        \node[anchor=west] at (5, 5) { \texttt{Entrada} };

        \node[anchor=west] at (6, 4) { \textbf{id} \code{apl}{+} \textbf{id} \code{apl}{×} \textbf{id} \texttt{\$} };
        \draw[-latex,thick] (7.05, 3.2) to (7.05, 3.8);

        \node[anchor=west] at (10, 5) { \texttt{Saída} };

        \node[anchor=west] at (11, 4) { $T\to FT'$ };
%        \node[anchor=west] at (11, 4) { $E'\to \code{apl}{+}TE'$ };
    \end{tikzpicture}

\end{frame}

\begin{frame}[fragile]{Comportamento do analisador preditivo não-recursivo para a cadeia ``$\textbf{id}\code{apl}{+}\textbf{id}\code{apl}{×}\textbf{id}$''}

    \begin{tikzpicture}
        \node[opacity=0] at (12, 0) { . };
        \node[opacity=0] at (12, 7) { . };

        \node at (2.5, 0.5) { \texttt{Pilha} };

        \draw[very thick] (1.5, 1) to (3.5, 1);

        \draw[thick] (2, 1) rectangle (3, 2);
        \node at (2.5, 1.5) { \texttt{\$} };

        \draw[thick] (2, 2) rectangle (3, 3);
        \node at (2.5, 2.5) { $E'$ };

        \draw[thick] (2, 3) rectangle (3, 4);
        \node at (2.5, 3.5) { $T'$ };

        \draw[thick] (2, 4) rectangle (3, 5);
        \node at (2.5, 4.5) { \textbf{id} };

        \node[anchor=west] at (5, 5) { \texttt{Entrada} };

        \node[anchor=west] at (6, 4) { \textbf{id} \code{apl}{+} \textbf{id} \code{apl}{×} \textbf{id} \texttt{\$} };
        \draw[-latex,thick] (7.05, 3.2) to (7.05, 3.8);

        \node[anchor=west] at (10, 5) { \texttt{Saída} };

        \node[anchor=west] at (11, 4) { $F\to \textbf{id}$ };
%        \node[anchor=west] at (11, 4) { $E'\to \code{apl}{+}TE'$ };
    \end{tikzpicture}

\end{frame}

\begin{frame}[fragile]{Comportamento do analisador preditivo não-recursivo para a cadeia ``$\textbf{id}\code{apl}{+}\textbf{id}\code{apl}{×}\textbf{id}$''}

    \begin{tikzpicture}
        \node[opacity=0] at (12, 0) { . };
        \node[opacity=0] at (12, 7) { . };

        \node at (2.5, 0.5) { \texttt{Pilha} };

        \draw[very thick] (1.5, 1) to (3.5, 1);

        \draw[thick] (2, 1) rectangle (3, 2);
        \node at (2.5, 1.5) { \texttt{\$} };

        \draw[thick] (2, 2) rectangle (3, 3);
        \node at (2.5, 2.5) { $E'$ };

        \draw[thick] (2, 3) rectangle (3, 4);
        \node at (2.5, 3.5) { $T'$ };

%        \draw[thick] (2, 4) rectangle (3, 5);
%        \node at (2.5, 4.5) { \textbf{id} };

        \node[anchor=west] at (5, 5) { \texttt{Entrada} };

        \node[anchor=west] at (6, 4) { \textbf{id} \code{apl}{+} \textbf{id} \code{apl}{×} \textbf{id} \texttt{\$} };
        \draw[-latex,thick] (7.45, 3.2) to (7.45, 3.8);

        \node[anchor=west] at (10, 5) { \texttt{Saída} };

%        \node[anchor=west] at (11, 4) { $F\to \textbf{id}$ };
%        \node[anchor=west] at (11, 4) { $E'\to \code{apl}{+}TE'$ };
    \end{tikzpicture}

\end{frame}


\begin{frame}[fragile]{Valores-L e valores-R}

    \begin{itemize}
        \item O significado de um identificador depende da posição onde ele ocorre em uma atribuição
        \pause

        \item No lado esquerdo, o identificador se refere à localização de memória onde o valor deve ser armazenado
        \pause

        \item No lado direito, o identificador se refere ao valor armazenado na localização de memória associada ao identificador
        \pause

        \item Valor-L e valor-R se referem aos valores apropriados para os lados esquerdo e direito de uma atribuição, respectivamente
        \pause

        \item Um mesmo identificador pode ser um valor-L e um valor-R na mesma atribuição (por exemplo, o identificador \code{cpp}{x} em \code{cpp}{x = x + 1})
    \end{itemize}

\end{frame}

\begin{frame}[fragile]{Manipulação da pilha}

    Uma máquina de pilha abstrata suporta as seguintes instruções para a manipulação da pilha:
    \begin{table}
        \center 
        \begin{tabularx}{0.9\textwidth}{p{3cm}X}
            \toprule
            \textbf{Instrução} & \textbf{Significado} \\
            \midrule
            \code{cpp}{push} $v$ & empilha $v$ \\
            \code{cpp}{pop} & desempilha o valor do topo da pilha \\
            \code{cpp}{valor-r} $p$ & empilha o valor armazenado no endereço de memória $p$ \\
            \code{cpp}{valor-l} $p$ & empilha o endereço de memória $p$ \\
            \code{cpp}{:=} & o valor-R do topo da pilha é armazenado no valor-L do subtopo (elemento que está abaixo do topo) da pilha \\
            \code{cpp}{copiar} & empilha o valor do topo da pilha \\
            \bottomrule
        \end{tabularx}
    \end{table}

\end{frame}
