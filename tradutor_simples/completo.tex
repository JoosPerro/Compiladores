\section{Código completo do tradutor}

\begin{frame}[fragile]{Código completo do tradutor}

    \begin{itemize}
        \item O código a seguir implementa um tradutor de expressões em forma infixa, terminadas por ponto-e-vírgula, para forma posfixa
        \pause

        \item Ele será implementado a partir da tradução dirigida a sintaxe
        \pause

        \item As expressões conterão números, identificadores e os operadores \code{cpp}{+, -, *, /, div} e \code{cpp}{mod}
        \pause

        \item Os tokens são: \textbf{id}, para identificadores, \textbf{num}, para constantes inteiras e \textbf{eof} para fim de arquivo
        \pause
        
        \item Cada módulo será implementado em um par de arquivos \code{cpp}{.cpp} e \code{cpp}{.h}, exceto pelo módulo principal (\code{cpp}{main.cpp})
    \end{itemize}

\end{frame}

\begin{frame}[fragile]{Especificação para um tradutor infixa-posfixa}

\[
    \begin{array}{lcll}
        inicio & \to & lista\ \mathbf{eof} \\
        lista & \to & expr\ \code{cpp}{;}\ \ \{\ imprimir(\mintinline{cpp}{'\n'})\ \}\ \ lista \\
        & | & \code{apl}{∊} \\
        expr & \to & expr\ \code{cpp}{+}\ termo & \{\ imprimir(\code{cpp}{'+'})\ \} \\
        & | & expr\ \code{cpp}{-}\ termo & \{\ imprimir(\code{cpp}{'-'})\ \} \\
        & | & termo \\
        termo & \to & termo\ \code{cpp}{*}\ fator & \{\ imprimir(\code{cpp}{'*'})\ \} \\
        & | & termo\ \code{cpp}{/}\ fator & \{\ imprimir(\code{cpp}{'/'})\ \} \\
        & | & termo\ \mathbf{div}\ fator & \{\ imprimir(\code{cpp}{"DIV"})\ \} \\
        & | & termo\ \mathbf{mod}\ fator & \{\ imprimir(\code{cpp}{"MOD"})\ \} \\
        & | & fator \\
        fator & \to & (expr) \\
        & | & \mathbf{id} & \{\ imprimir(\mathbf{id}.lexema)\ \} \\
        & | & \mathbf{num} & \{\ imprimir(\mathbf{id}.valor)\ \}
    \end{array}
\]

\end{frame}

\begin{frame}[fragile]{Descrição dos tokens}

    \begin{table}
        \centering 

        \begin{tabularx}{0.95\textwidth}{Xcl}
        \toprule
        \textbf{Lexema} & \textbf{Token} & \textbf{Atributo} \\
        \midrule
        espaço em branco \\
        \rowcolor[gray]{0.9}
        sequência de dígitos & \code{cpp}{NUM} & valor númerico da sequência \\
        \textbf{div} & \code{cpp}{DIV} \\
        \rowcolor[gray]{0.9}
        \textbf{mod} & \code{cpp}{MOD} & \\
        sequências iniciada em letra e seguida de letras e dígitos & \code{cpp}{ID} & lexema \\
        \rowcolor[gray]{0.9}
        caractere de fim de arquivo & \code{cpp}{DONE} & \\
        qualquer outro caractere & o próprio caractere & \\
        \bottomrule
        \end{tabularx}

    \end{table}

\end{frame}

\begin{frame}[fragile]{Módulo \code{cpp}{main.cpp}}

    \begin{itemize}
        \item Este módulo é responsável pela início da execução do programa
        \pause

        \item Ele simplesmente invoca o tradutor
    \end{itemize}
    \pause

    \inputcode{cpp}{codes/tradutor/main.cpp}

\end{frame}

\begin{frame}[fragile]{Módulo \code{cpp}{token}}

    \begin{itemize}
        \item Este módulo define uma estrutura para a representação de um token
        \pause

        \item Cada token tem um tipo e um atributo associado
        \pause

        \item Como o atributo de \code{cpp}{NUM} é inteiro e de \code{cpp}{ID} é string, foi utilizado o tipo \code{cpp}{variant<int, string>} de C++17
        \pause

        \item Os tipos dos tokens foram codificados como inteiros, com valores foram da faixa ASCII, para evitar conflitos com os tokens compostos por um
            único caractere
        \pause

        \item Também foi definida uma função para a impressão de um token, que trata internamente as diferenças entre os tipos
    \end{itemize}

\end{frame}

\begin{frame}[fragile]{Arquivo \code{cpp}{token.h}}
    \inputcode{cpp}{codes/tradutor/token.h}
\end{frame}

\begin{frame}[fragile]{Arquivo \code{cpp}{token.cpp}}
    \inputsnippet{cpp}{1}{18}{codes/tradutor/token.cpp}
\end{frame}

\begin{frame}[fragile]{Arquivo \code{cpp}{token.cpp}}
    \inputsnippet{cpp}{20}{38}{codes/tradutor/token.cpp}
\end{frame}

\begin{frame}[fragile]{Tabela de símbolos}

    \begin{itemize}
        \item Este módulo define uma classe para a representação de uma tabela de símbolos
        \pause

        \item Esta classe segue o padrão Singleton, pois deve haver uma única tabela de símbolos, a qual será compartilhada por todas as fases do compilador
        \pause

        \item A estrutura que armazena os símbolos é um dicionário (classe \code{cpp}{map} de C++), onde as chaves são os lexemas e os valores são os tokens
        \pause

        \item Na inicialização da tabela são adicionadas, ao dicionário, todas as palavras reservadas
        \pause

        \item O método \code{cpp}{insert()} insere um novo símbolo no dicionário
        \pause

        \item O método \code{cpp}{find()} localizar um símbolo já inserido, ou retorna vazio, caso o dicionário não tenha nenhum token associado ao lexema
            passado como parâmetro
    \end{itemize}

\end{frame}

\begin{frame}[fragile]{Arquivo \code{cpp}{table.h}}
    \inputcode{cpp}{codes/tradutor/table.h}
\end{frame}

\begin{frame}[fragile]{Arquivo \code{cpp}{table.cpp}}
    \inputsnippet{cpp}{1}{19}{codes/tradutor/table.cpp}
\end{frame}

\begin{frame}[fragile]{Arquivo \code{cpp}{table.cpp}}
    \inputsnippet{cpp}{21}{39}{codes/tradutor/table.cpp}
\end{frame}

\begin{frame}[fragile]{Módulo \code{cpp}{scanner}}

    \begin{itemize}
        \item Este módulo implementa o analisador léxico do tradutor
        \pause

        \item Como este analisador não tem estado, ele foi implementado por meio de um \code{cpp}{namespace}, o que permite usar a mesma notação de método
            estático de um classe, embora a implementação seja a de uma função regular de C++
        \pause

        \item A função \code{cpp}{next_token()} extrai o próximo token da entrada padrão
        \pause

        \item O \textit{scanner} ignora todos os espaços em branco
        \pause

        \item Os demais tokens são extraídos conforme a especificação
    \end{itemize}

\end{frame}

\begin{frame}[fragile]{Arquivo \code{cpp}{scanner.h}}
    \inputcode{cpp}{codes/tradutor/scanner.h}
\end{frame}

\begin{frame}[fragile]{Arquivo \code{cpp}{scanner.cpp}}
    \inputsnippet{cpp}{1}{19}{codes/tradutor/scanner.cpp}
\end{frame}

\begin{frame}[fragile]{Arquivo \code{cpp}{scanner.cpp}}
    \inputsnippet{cpp}{21}{37}{codes/tradutor/scanner.cpp}
\end{frame}

\begin{frame}[fragile]{Arquivo \code{cpp}{scanner.cpp}}
    \inputsnippet{cpp}{39}{57}{codes/tradutor/scanner.cpp}
\end{frame}

\begin{frame}[fragile]{Módulo \code{cpp}{parser}}

    \begin{itemize}
        \item Este módulo implementa o analisador sintático do tradutor
        \pause

        \item De fato, é um analisador gramatical preditivo que, em conjunto com as ações semânticas especificadas, produz um tradutor de notação infixa para
            posfixa
        \pause

        \item Ele invoca o \textit{scanner} para obter os tokens da entrada, um por vez
        \pause

        \item Cada não-terminal da gramática é implementado por meio de um procedimento
        \pause

        \item Cada expressão da entrada será traduzida para uma linha da saída, em notação posfixa
        \pause

        \item As expressões da entrada devem ser terminadas por \code{cpp}{;}
    \end{itemize}

\end{frame}

\begin{frame}[fragile]{Arquivo \code{cpp}{parser.h}}
    \inputcode{cpp}{codes/tradutor/parser.h}
\end{frame}

\begin{frame}[fragile]{Arquivo \code{cpp}{parser.cpp}}
    \inputsnippet{cpp}{1}{15}{codes/tradutor/parser.cpp}
\end{frame}

\begin{frame}[fragile]{Arquivo \code{cpp}{parser.cpp}}
    \inputsnippet{cpp}{17}{31}{codes/tradutor/parser.cpp}
\end{frame}

\begin{frame}[fragile]{Arquivo \code{cpp}{parser.cpp}}
    \inputsnippet{cpp}{33}{50}{codes/tradutor/parser.cpp}
\end{frame}

\begin{frame}[fragile]{Arquivo \code{cpp}{parser.cpp}}
    \inputsnippet{cpp}{52}{70}{codes/tradutor/parser.cpp}
\end{frame}

\begin{frame}[fragile]{Arquivo \code{cpp}{parser.cpp}}
    \inputsnippet{cpp}{72}{90}{codes/tradutor/parser.cpp}
\end{frame}

\begin{frame}[fragile]{Arquivo \code{cpp}{parser.cpp}}
    \inputsnippet{cpp}{92}{108}{codes/tradutor/parser.cpp}
\end{frame}

\begin{frame}[fragile]{Arquivo \code{cpp}{parser.cpp}}
    \inputsnippet{cpp}{110}{128}{codes/tradutor/parser.cpp}
\end{frame}

\begin{frame}[fragile]{Módulo \code{cpp}{error}}

    \begin{itemize}
        \item Este módulo é responsável pelo tratamento de erros
        \pause

        \item A abordagem utilizada é simplificada: é impressa a mensagem indicada e o programa é encerrado por meio da função \code{cpp}{exit()}
        \pause

    \end{itemize}

    \inputcode{cpp}{codes/tradutor/error.cpp}
\end{frame}
