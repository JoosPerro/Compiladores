\section{Um tradutor para expressões simples}

\begin{frame}[fragile]{Tradutor dirigido pela sintaxe}

    \begin{itemize}
        \item Quando não há associação de atributos aos não-terminais, um tradutor dirigido pela sintaxe pode ser construído a partir da extensão de um analisador 
        gramatical preditivo
        \pause

        \item Para isso, inicialmente construa o analisador gramatical preditivo
        \pause

        \item Em seguida, copie as ações sintáticas do tradutor nas posições adequadas no analisador gramatical preditivo
        \pause

        \item Se a gramática tiver uma ou mais produções recursivas à esquerda, é preciso modificar a gramatica para eliminar esta recursão antes de proceder com
            a construção do analisador gramatical preditivo
    \end{itemize}

\end{frame}

\begin{frame}[fragile]{Transformação de produções recursivas à esquerda }

    \begin{block}{Transformação de produção recursiva à esquerda}
        Seja $A \to A\alpha\ |\ A\beta\ |\ \gamma$ uma produção recursiva à esquerda. Esta produção equivale às produções recursivas à direta
        \[
            \begin{array}{l}
                A \to \gamma R \\
                R \to \alpha R\ |\ \beta R\ |\ \code{apl}{∊} \\
            \end{array}
        \]
        onde $\alpha$ e $\beta$ é uma cadeia de terminais e não-terminais que não começam com $A$ e nem terminam com $R$.
    \end{block}

\end{frame}

\begin{frame}[fragile]{Exemplo de transformação de produção recursiva à esquerda}

    \begin{figure}
        \centering

        \begin{tikzpicture} 
            \node[anchor=west] (A1) at (0, 8.0) { \footnotesize $expr \to expr + digito$ };
            \node[anchor=west] (A2) at (0, 7.5) { \footnotesize $expr \to expr - digito$ };
            \node[anchor=west] (A3) at (0, 7.0) { \footnotesize $expr \to digito$ };

            \pause

            \draw[thick,-latex] (1, 6.7) to (1, 5.3);
            \node[anchor=west] (B1) at (0, 5.0) { \footnotesize $A = expr$ };
            \node[anchor=west] (B2) at (0, 4.5) { \footnotesize $\alpha = +\ digito$ };
            \node[anchor=west] (B3) at (0, 4.0) { \footnotesize $\beta = -\ digito$ };
            \node[anchor=west] (B4) at (0, 3.5) { \footnotesize $\gamma = digito$ };
            \node[anchor=west] (B5) at (0, 3.0) { \footnotesize $R = resto$ };

            \pause
            \node[anchor=west] (C1) at (5, 4.75) { \footnotesize $expr \to digito\ resto$ };
            \node[anchor=west] (C2) at (5, 4.25) { \footnotesize $resto \to +\ digito\ resto$ };
            \node[anchor=west] (C3) at (5, 3.75) { \footnotesize $resto \to -\ digito\ resto$ };
            \node[anchor=west] (C4) at (5, 3.25) { \footnotesize $resto \to$ \code{apl}{∊} };

            \draw[thick,-latex] (2.3, 4) to (4.7, 4);
        \end{tikzpicture} 
    \end{figure}

\end{frame}

\begin{frame}[fragile]{Esquema de tradução da gramática para expressões para a notação posfixa}

    \[
        \begin{array}{rl}
            expr \to & digito\ resto \\
            \\
            resto \to & +\ digito\ \ \{ imprimir(\code{cpp}{'+'}) \}\ \ resto \\
            resto \to & -\ digito\ \ \{ imprimir(\code{cpp}{'-'}) \}\ \ resto \\
            resto \to & \code{apl}{∊} \\
            \\
            digito \to & 0\ \ \{ imprimir(\code{cpp}{'0')} \} \\
            digito \to & 1\ \ \{ imprimir(\code{cpp}{'1')} \} \\
            \ldots \\
            digito \to & 9\ \ \ \ \{ imprimir(\code{cpp}{'9')} \} \\
        \end{array}
    \]
\end{frame}

\begin{frame}[fragile]{Rotina de extração do próximo token em C++}

    \inputsnippet{cpp}{1}{10}{codes/tradutor.cpp}

\end{frame}

\begin{frame}[fragile]{Rotina de tratamento de erro e declaração de {\it lookahead} em C++}

    \inputsnippet{cpp}{12}{18}{codes/tradutor.cpp}

\end{frame}

\begin{frame}[fragile]{Rotina de reconhecimento de tokens em C++}

    \inputsnippet{cpp}{20}{26}{codes/tradutor.cpp}

\end{frame}

\begin{frame}[fragile]{Rotina associada ao não-terminal {\it expr} em C++}

    \inputsnippet{cpp}{52}{56}{codes/tradutor.cpp}

\end{frame}

\begin{frame}[fragile]{Rotina associada ao não-terminal {\it digito} em C++}

    \inputsnippet{cpp}{28}{38}{codes/tradutor.cpp}

\end{frame}

\begin{frame}[fragile]{Rotina associada ao não-terminal {\it resto} em C++}

    \inputsnippet{cpp}{40}{50}{codes/tradutor.cpp}

\end{frame}

\begin{frame}[fragile]{Rotina principal do tradutor em C++}

    \inputsnippet{cpp}{58}{67}{codes/tradutor.cpp}

\end{frame}
